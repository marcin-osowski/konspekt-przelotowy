\documentclass{article}
\usepackage[utf8]{inputenc}
\usepackage{polski}
\usepackage{newlfont}
\title{Pierwsze przeloty szybowcowe}
\author{Osowski~Marcin\\Aeroklub~Warmińsko-Mazurski}
\begin{document}

\maketitle
\newpage

\begin{abstract}
W tym opracowaniu zebrałem garść porad dla pilotów szybowcowych
pragnących opuścić stożek dolotowy i rozpocząć
przygodę przelotową. Informacje tu zawarte bazują na moim
własnym doświadczeniu i cudzych opracowaniach. Treść została
podzielona na trzy części: przygotowanie do przelotu, taktyka przelotowa, 
lądowanie w terenie przygodnym i postępowanie po lądowaniu.
\end{abstract}
\newpage

\begin{center}\begin{huge}
Zrzeczenie się odpowiedzialności
\end{huge}\end{center}
[TODO]
%Informacje zawarte w niniejszym opracowaniu są podane 
%
%Pomimo starań autora informacje zawarte w tym opracowaniu nie mogą służyć
%jako instrukcja wykonywania bezpiecznych lotów szybowcowych. 
\newpage

\tableofcontents
\newpage

\section{Przygotowania do przelotu}
A więc zdecydowałeś się/zdecydowałaś się wykonać przelot szybowcowy, być może
pierwszy w Twoim życiu. Zacznijmy od wypunktowania tego co należy
przygotować wcześniej.

\subsection{Lista rzeczy do zrobienia wcześniej}
Potrzebujesz:
\begin{enumerate}
\item aeroklubowej zgody na przelot szybowcem
\item nawigacji
\item przygotowanego wózka transportowego
\item (do lotów warunkowych) rejestratora
\end{enumerate}

\subsubsection{Aeroklubowa zgoda na wykonanie przelotu}
Tradycyjnie w Aeroklubie Warmińsko-Mazurskim przed jakimkolwiek przelotem
wymaga się od uczniów-pilotów zdobycia przewyższenia 1000 m oraz wykonania
co najmniej pięcio-godzinnego lotu szybowcem. Dodatkowo, każdy pilot musi
przygotować się do lądowania poza lotniskiem -- oznacza to wykonanie co
najmniej dziesięciu lądowań na typie szybowca na którym planowany jest
przelot.

Pamiętaj, że warunki pogodowe i stan upraw na polach mogą cię zaskoczyć,
dlatego zawsze konsultuj swoje plany z Szefem Wyszkolenia oraz innymi
doświadczonymi pilotami.

\subsubsection{Nawigacja}
Po pierwsze: mapa. Jak należy przygotować mapę?
\newpage

\section{Taktyka przelotowa}
\newpage

\section{Lądowanie w terenie przygodnym}


\end{document} 
