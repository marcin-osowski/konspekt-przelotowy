\documentclass{article}
\usepackage[utf8]{inputenc}
\usepackage{polski}
\usepackage{newlfont}
\usepackage{hyperref}
\title{Pierwsze przeloty szybowcowe}
\author{Osowski~Marcin\\Aeroklub~Warmińsko-Mazurski}
\hypersetup{
    pdftitle={Pierwsze przeloty szybowcowe},    % title
    pdfauthor={Osowski Marcin},     % author
    colorlinks=true,          % false: boxed links; true: colored links
    linkcolor=black,          % color of internal links
    citecolor=black,          % color of links to bibliography
    filecolor=black,          % color of file links
    urlcolor=black            % color of external links
}

\begin{document}

\maketitle
\newpage

\begin{abstract}
W tym opracowaniu zebrałem garść informacji dla pilotów szybowcowych
pragnących opuścić stożek dolotowy i rozpocząć
przygodę przelotową. Informacje tu zawarte bazują na moim
własnym doświadczeniu i cudzych opiniach, których autorstwo jest na ogół
trudne do ustalenia. Treść została podzielona na trzy części:
przygotowanie do przelotu, taktyka przelotowa, 
lądowanie w terenie przygodnym i postępowanie po lądowaniu.
\end{abstract}
\newpage

\begin{center}\begin{huge}
Zrzeczenie się odpowiedzialności
\end{huge}\end{center}
Informacje zawarte w niniejszym opracowaniu oparte są o doświadczenie
autora i inne tego typu opracowania; mają znaczenie wyłącznie informacyjne
i edukacyjne. Pomimo dołożenia wielu starań nie mogą być stosowane
jako porady i nie mogą być podstawą roszczeń wobec autora,
Aeroklubu Warmińsko-Mazurskiego ani kogokolwiek rozprzestrzeniającego
ten dokument.
\newpage

\tableofcontents
\newpage

\section{Przygotowania do przelotu}
A więc zdecydowałeś się/zdecydowałaś się wykonać przelot szybowcowy, być może
pierwszy w Twoim życiu. Zacznijmy od wypunktowania tego co należy
przygotować wcześniej.

\subsection{Lista rzeczy do zrobienia wcześniej}
Potrzebujesz:
\begin{enumerate}
\item aeroklubowej zgody na przelot szybowcem
\item pomocy nawigacyjnych
\item przygotowanego wózka transportowego
\item (do lotów warunkowych) rejestratora
%TODO - dodać przygotowanie trasy do pierwszego przelotu
% na dużo przed dniem przelotu
\end{enumerate}

\subsubsection{Aeroklubowa zgoda na wykonanie przelotu}
Tradycyjnie w Aeroklubie Warmińsko-Mazurskim przed jakimkolwiek przelotem
wymaga się od uczniów-pilotów zdobycia przewyższenia 1000 m oraz wykonania
co najmniej pięcio-godzinnego lotu szybowcem. Dodatkowo, każdy pilot musi
przygotować się do lądowania poza lotniskiem -- oznacza to wykonanie co
najmniej dziesięciu lądowań na typie szybowca na którym planowany jest
przelot.

Pamiętaj, że warunki pogodowe i stan upraw na polach mogą cię zaskoczyć,
dlatego zawsze konsultuj swoje plany z Szefem Wyszkolenia oraz innymi
doświadczonymi pilotami.

\subsubsection{Pomoce nawigacyjne}
Musisz mieć mapę. Jak należy przygotować mapę?
[TODO] % TODO

Popularną pomocą nawigacyjną jest PDA (nazwa zwyczajowa: ,,lusterko'') 
wykonujący program przystosowany
do potrzeb szybowników. Na rynku jest dostępnych co najmniej 5
rozwiązań:
\begin{enumerate}
\item GPS\_LOG -- darmowy
\item LK8000 -- damowy
\item SeeYou Mobile
\item WinPilot
\item XCSoar -- darmowy
\end{enumerate}
Moje doświadczenia ograniczają się do punktu 2. i punktu 4., ale
opinie o każdym z wyżej wymienionych programów są pozytywne.

\subsubsection{Przygotowany wózek transportowy}
Należy liczyć się z lądowaniem poza lotniskiem. 
Wózek transportowy musi być sprawny i posiadać wszystkie wymagane
dokumenty (świadectwo rejestracji, ważne ubezpieczenie OC). Dobrze jest
umówić się z kimś co do ewentualnego transportu szybowca z pola -- po
lądowaniu zazwyczaj mamy mniejsze możliwości uzgadniania takich kwestii.

Dobrym zwyczajem jest zabieranie ze sobą do szybowca kliku metrów
linki transportowej. Zdarza się, że lądujemy na polu w pobliżu którego
rolnicy wykonują jakieś prace -- można wtedy poprosić ich o ściągnięcie
szybowca na skraj pola.

\subsubsection{Rejestrator (do lotów warunkowych)}
[TODO] % TODO

\subsection{W dniu przelotu}
Jeżeli postąpiłeś/postąpiłaś zgodnie z powyżej wymienionymi wskazówkami to
zostaje niewiele do zrobienia. W zasadzie działania można ograniczyć do:
\begin{enumerate}
\item przeczytania prognozy pogody
\item sprawdzenia zajętości stref powietrznych
\item wytyczenia trasy
\item odejścia na trasę :)
\end{enumerate}

\subsubsection{Prognoza pogody}
Jest bardzo wiele modeli pogodowych użytecznych dla szybowników. Wymienię
te, których sam używałem i na temat których mam jakieś zdanie.
\begin{enumerate}
\item \url{http://rasp.linta.de/GERMANY/index_en.html} -- model napisany
przez szybownika-meteorologa (Dr. John W. Glendening) dla szybowników.
Obliczany co 24 godziny (nocą) na 48 godzin do przodu. Moim skromnym
zdaniem jest to znakomite narzędzie do planowania trasy przelotu.
Jeżeli nie chcesz wgłębiać się w zbyt wiele szczegółów, to najważniejszymi
parametrami są ,,Thermal Updraft Velocity and B/S ratio'' (siła noszeń) oraz
,,Cu Cloudbase where Cu Potential $>$ 0'' (podstawy chmur tam, gdzie chmur
się spodziewamy).

\item \url{http://sat24.com/pl} -- zdjęcia satelitarne w paśmie widzialnym~i
podczerwieni, robione co 15 minut. Niezastąpiony wskaźnik rozwoju termiki
i frontów atmosferycznych.

\item \url{http://new.meteo.pl/} -- dwa modele: COAMPS i UM. Model COAMPS
niestety słabo wpasowywuje się w pogodę, natomiast UM jest znakomity do
wglądu krótkoterminowego (obliczany jest co 6 godzin na 48 godzin do przodu).

\item \url{http://ows.public.sembach.af.mil/index.cfm?section=SFCProg} -- 
prognoza przygotowywana przez NATO. Bardzo skuteczna, natomiast nie
skupia się na wszystkich zjawiskach (przewiduje tylko układ ciśnień i
schemat frontów atmosferycznych).

\item \url{http://www.windguru.cz/pl/} -- model GFS, jeden z wielu.
Prognoza długoterminowa.

\item \url{http://www.imgw.pl/} -- zakładka ,,Awiacja''. Prognoza lotnicza.
Wymaga znajomości specjalistycznych skrótów.

\end{enumerate}
\subsubsection{Zajętość stref powietrznych}
Do sprawdzenia na stronie \url{http://amc.pata.pl/}, zakładka
,,AUP bieżący'' lub na mapie ,,Elementy przestrzeni/AUP''. Można też
telefonicznie, numery do odnalezienia na stronie AMC PATA.
\subsubsection{Wytyczanie trasy}
Niezastąpiona staje się mapa z zaznaczonymi punktami zwrotnymi:

[TODO] % TODO: dodać zdjęcie mapy punktów zwrotnych
\noindent
Dobrze jest w tym momencie zasięgnąć opinii doświadczonych szybowników.

\newpage

\section{Taktyka przelotowa}
[TODO] % TODO
\newpage

\section{Lądowanie w terenie przygodnym}
[TODO] % TODO

\end{document} 
