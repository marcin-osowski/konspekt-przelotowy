\documentclass{article}
\usepackage{amsfonts}
\usepackage{amssymb}
\usepackage[utf8]{inputenc}
\usepackage[fleqn]{amsmath}
\usepackage{polski}
\usepackage{amsthm}
\usepackage{newlfont}
\title{Pierwsze przeloty szybowcowe}
\author{Osowski~Marcin\\Aeroklub~Warmińsko-Mazurski}
\begin{document}

\maketitle
\newpage

\begin{abstract}
W tym opracowaniu zebrałem garść porad dla pilotów szybowcowych
(także uczniów pilotów) pragnących opuścić stożek dolotowy i rozpocząć
przygodę przelotową. Informacje tu zawarte bazują na moim
własnym doświadczeniu i cudzych opracowaniach. Treść została
podzielona na trzy części: przygotowanie do przelotu, taktyka przelotowa, 
lądowanie w terenie przygodnym i postępowanie po lądowaniu.
\end{abstract}
\newpage

\begin{center}\begin{huge}
Zrzeczenie się odpowiedzialności
\end{huge}\end{center}
[TODO]
%Informacje zawarte w niniejszym opracowaniu są podane 
%
%Pomimo starań autora informacje zawarte w tym opracowaniu nie mogą służyć
%jako instrukcja wykonywania bezpiecznych lotów szybowcowych. 
\newpage

\tableofcontents
\newpage

\section{Przygotowania do przelotu}
\newpage

\section{Taktyka przelotowa}
\newpage

\section{Lądowanie w terenie przygodnym}


\end{document} 
