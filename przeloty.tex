\documentclass{article}
\usepackage[utf8]{inputenc}
\usepackage{polski}
\usepackage{newlfont}
\title{Pierwsze przeloty szybowcowe}
\author{Osowski~Marcin\\Aeroklub~Warmińsko-Mazurski}
\begin{document}

\maketitle
\newpage

\begin{abstract}
W tym opracowaniu zebrałem garść porad dla pilotów szybowcowych
pragnących opuścić stożek dolotowy i rozpocząć
przygodę przelotową. Informacje tu zawarte bazują na moim
własnym doświadczeniu i cudzych opracowaniach. Treść została
podzielona na trzy części: przygotowanie do przelotu, taktyka przelotowa, 
lądowanie w terenie przygodnym i postępowanie po lądowaniu.
\end{abstract}
\newpage

\begin{center}\begin{huge}
Zrzeczenie się odpowiedzialności
\end{huge}\end{center}
Informacje zawarte w niniejszym opracowaniu oparte są o doświadczenie
autora i inne tego typu opracowania; mają znaczenie wyłącznie informacyjne
i edukacyjne. Pomimo dołożenia wielu starań nie mogą być stosowane
jako porady i nie mogą być podstawą roszczeń wobec autora,
Aeroklubu Warmińsko-Mazurskiego ani kogokolwiek rozprzestrzeniającego
ten dokument.
\newpage

\tableofcontents
\newpage

\section{Przygotowania do przelotu}
A więc zdecydowałeś się/zdecydowałaś się wykonać przelot szybowcowy, być może
pierwszy w Twoim życiu. Zacznijmy od wypunktowania tego co należy
przygotować wcześniej.

\subsection{Lista rzeczy do zrobienia wcześniej}
Potrzebujesz:
\begin{enumerate}
\item aeroklubowej zgody na przelot szybowcem
\item pomocy nawigacyjnych
\item przygotowanego wózka transportowego
\item (do lotów warunkowych) rejestratora
\end{enumerate}

\subsubsection{Aeroklubowa zgoda na wykonanie przelotu}
Tradycyjnie w Aeroklubie Warmińsko-Mazurskim przed jakimkolwiek przelotem
wymaga się od uczniów-pilotów zdobycia przewyższenia 1000 m oraz wykonania
co najmniej pięcio-godzinnego lotu szybowcem. Dodatkowo, każdy pilot musi
przygotować się do lądowania poza lotniskiem -- oznacza to wykonanie co
najmniej dziesięciu lądowań na typie szybowca na którym planowany jest
przelot.

Pamiętaj, że warunki pogodowe i stan upraw na polach mogą cię zaskoczyć,
dlatego zawsze konsultuj swoje plany z Szefem Wyszkolenia oraz innymi
doświadczonymi pilotami.

\subsubsection{Pomoce nawigacyjne}
Musisz mieć mapę. Jak należy przygotować mapę?
[TODO] % TODO

Popularną pomocą nawigacyjną jest PDA (nazwa zwyczajowa: ,,lusterko'') 
wykonujący program przystosowany
do potrzeb szybowników. Na rynku jest dostępnych co najmniej 5
rozwiązań:
\begin{enumerate}
\item GPS\_LOG -- darmowy
\item LK8000 -- damowy
\item SeeYou Mobile
\item WinPilot
\item XCSoar -- darmowy
\end{enumerate}
Moje doświadczenia ograniczają się do punktu 2. i punktu 4., ale
opinie o każdym z wyżej wymienionych programów są pozytywne.

\subsubsection{Przygotowany wózek transportowy}
Należy liczyć się z lądowaniem poza lotniskiem. 
Wózek transportowy musi być sprawny i posiadać wszystkie wymagane
dokumenty (świadectwo rejestracji, ważne ubezpieczenie OC). Dobrze jest
umówić się z kimś co do ewentualnego transportu szybowca z pola -- po
lądowaniu zazwyczaj mamy mniejsze możliwości uzgadniania takich kwestii.

Dobrym zwyczajem jest zabieranie ze sobą do szybowca kliku metrów
linki transportowej. Zdarza się, że lądujemy na polu w pobliżu którego
rolnicy wykonują jakieś prace -- można wtedy poprosić ich o ściągnięcie
szybowca na skraj pola.

\subsubsection{Rejestrator (do lotów warunkowych)}
[TODO] % TODO
\newpage

\section{Taktyka przelotowa}
[TODO] % TODO
\newpage

\section{Lądowanie w terenie przygodnym}
[TODO] % TODO

\end{document} 
